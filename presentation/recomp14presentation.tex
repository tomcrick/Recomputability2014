\documentclass[pdftex]{beamer}
\usepackage[british]{babel}
\usepackage{graphicx}
\usepackage{url}
\usepackage[normalem]{ulem}
\usepackage{tikz}
\usetikzlibrary{mindmap,trees}

\mode<presentation>
\usetheme{Warsaw}
\useoutertheme{infolines}

\setbeamertemplate{navigation symbols}{}
% \usebeamertemplate*{logo}
% \logo{\includegraphics[height=0.85cm]{cmetlogo.png}}


\title[Recomputability 2014]{``Share and Enjoy'': Publishing Useful and Usable Scientific Models}
\author[\#recomp14]{Tom Crick$^1$, Benjamin A. Hall$^2$, Samin Ishtiaq$^3$ and Kenji Takeda$^3$}
\institute[@DrTomCrick]{$^1$ Department of Computing, Cardiff
  Metropolitan University\\$^2$ MRC Cancer Unit, University of
  Cambridge\\$^3$ Microsoft Research Cambridge}
\date{11 December 2014}

\begin{document}

% titlepage
\begin{frame}
\titlepage
\end{frame}

% TOC
% \section*{Talk Outline} 
% \begin{frame} 
% \tableofcontents 
% \end{frame} 

\section{Motivation}

\begin{frame}
\frametitle{Our Computational World}
%\begin{alertblock}{The future?}
{\Large{{\emph{``[Computational techniques] have moved on from assisting scientists in 
doing science, to transforming both how science is 
done and what science is done.''}}}}\\
\flushright{{\emph{Science as an open enterprise}, Royal Society (June 2012)}\\
{\scriptsize{\url{https://royalsociety.org/policy/projects/science-public-enterprise/}}}}
%\end{alertblock}
\end{frame}


{ % all template changes are local to this group.
    \setbeamertemplate{navigation symbols}{}
    \begin{frame}[plain]
        \begin{tikzpicture}[remember picture,overlay]
            \node[at=(current page.center)] {
                \includegraphics[width=0.9\paperwidth]{overlyhonesttweet.png}
            };
        \end{tikzpicture}
     \end{frame}
}

{ % all template changes are local to this group.
    \setbeamertemplate{navigation symbols}{}
    \begin{frame}[plain]
        \begin{tikzpicture}[remember picture,overlay]
            \node[at=(current page.center)] {
                \includegraphics[width=0.9\paperwidth]{phd031214s.png}
            };
        \end{tikzpicture}
     \end{frame}
}

\section{Models, Algorithms and Benchmarks}

\begin{frame}
\frametitle{Sharing}
Two key types of results arise from work done in the computational sciences:
\begin{itemize}
\item Models
\item Algorithms
\end{itemize}
\vspace{1em}
Fundamental advantage of computer science and more broadly,
computational science: {\textbf{the unique ability to share the raw outputs of
their research as software and datafiles.}}
\end{frame}

\begin{frame}
\frametitle{Models, Algorithms and Benchmarks}
\begin{itemize}
\item Abstraction levels ({\scriptsize{abstract vs. concrete}})
\item Benchmark repositories\\({\scriptsize{e.g. UCI Machine Learning Repository, Netflix Prize benchmarks, SMT Competition, SV-COMP, Answer Set Programming Competition and the Termination Problem Database.}})
\item Protocols as scripts ({\scriptsize{workflow reproducibility e.g. molecular dynamics}})
\item Performance and scalability ({\scriptsize{is performance a key issue?}})
\end{itemize}
\pause
\vspace{1em}{\textbf{We propose to develop a prototype open software platform which will automate reproducibility for algorithms and models.}}
\end{frame}


\section{A System for Automating Reproducibility}

\begin{frame}
\frametitle{A System for Automating Reproducibility in Science}
\begin{itemize}
\item Linking open software, algorithms and models
\item Open and community-curated benchmarks
\item Integrated continuous integration system: authoritative source of results for these algorithms running on these benchmarks.
\end{itemize}
\end{frame}

\begin{frame}
\frametitle{Proposed Workflow}
\begin{center}
\includegraphics[width=0.9\paperwidth]{workflow.png}
\end{center}
\end{frame}

\section{Future}

\begin{frame}
\frametitle{A System for Automating Reproducibility in Science}
{\small{\begin{itemize}
\item Build a cloud service which automatically pulls and compiles code from source repos;
\item Run automated tests defined by the developers on the code;
\item Perform analysis of benchmark sets supplied by both the developer and external users;
\item Provide persistent audit trails for software and benchmarks
  results;
\item Collaborate with key stakeholders in the open software/open
  data/open access/open science space, as well as key e-infrastructure
  organisations e.g. GitHub, figshare, SSI, Mozilla Science Lab,
  Digital Science, etc.
\item {\textbf{Key:}} engage with key communities to embed system/workflow and effect
  cultural change.
\end{itemize}}}
\end{frame}


\section{Acknowledgements}

\begin{frame}
\frametitle{Acknowledgements}
\begin{center}
\includegraphics[width=0.9\paperwidth]{ssi.png}
\end{center}
\end{frame}


\section{References}

\begin{frame}
\frametitle{References}
{\small{\begin{itemize}
\item Tom Crick, Benjamin A. Hall, Samin Ishtiaq and Kenji
  Takeda. {\emph{``Share and Enjoy'': Publishing Useful and Usable
      Scientific Models}}. In 1st International Workshop on
  Recomputability, 2014: \url{http://arxiv.org/abs/1409.0367}
\item Tom Crick, Benjamin A. Hall and Samin Ishtiaq. {\emph{``Can I Implement
  Your Algorithm?'': A Model for Reproducible Research Software}}. In
  Proceedings of 2nd International Workshop on Sustainable Software
  for Science: Practice and Experiences (WSSSPE2), 2014:
  \url{http://arxiv.org/abs/1407.5981}
\item Digital Science Catalyst Grant (Nov 2014):
  \url{https://github.com/tomcrick/DSCatalyst} 
\item Microsoft Azure for Research Grant (Dec 2014):
  \url{https://github.com/tomcrick/Azure4Research} 
\end{itemize}}}
\end{frame}

\end{document}
